\documentclass{report}
\usepackage{ragged2e}


\begin{document}
    \begin{titlepage}
        \vspace*{5cm}
        \begin{center}
            \Huge{\textbf{Estructuras de Datos \\ Proyecto 1}}
        \end{center}
        \vspace*{2cm}
        \begin{center}
            \textbf{Integrantes}
            \begin{itemize}
                \item Joaquín Andrés León Ulloa
                \item Richard Alejandro Gonzales Lara
            \end{itemize}
            \vspace*{5mm}
            \textbf{Profesor}
            \begin{itemize}
                \item Diego Seco Navarro
            \end{itemize}
        \end{center}
    \end{titlepage}
    \section*{Descripción de la Tarea}
        \justify
        La tarea consiste en implementar un simple
        sistema de predicción de texto, ya sea en
        español o en inglés.
        
        \justify
        El programa toma un diccionario con posibles
        palabras, y un prefijo correspondiente a la
        palabra que se quiere predecir. Además, toma
        un número $k$ que corresponde al número de
        predicciones que se quiere sacar del diccionario.
        
        \justify
        Los archivos con cadenas a buscar, y donde se
        debe escribir el output del programa, también
        se pasan como argumentos
        
        \begin{center}
            La invocación del programa es de la
            siguiente manera:\\
            
            \textbf{predictor diccionario.txt
            cadenas.txt resultados.txt k}
        \end{center}
        
        \section*{Cómo Funciona?}
        
        
        \section*{Gráficos}
            \subsection*{Uso de memoria}
            \subsection*{Tiempo de ejecución}
\end{document}